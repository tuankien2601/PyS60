% Copyright (c) 2005-2009 Nokia Corporation
%
% Licensed under the Apache License, Version 2.0 (the "License");
% you may not use this file except in compliance with the License.
% You may obtain a copy of the License at
%
%     http://www.apache.org/licenses/LICENSE-2.0
%
% Unless required by applicable law or agreed to in writing, software
% distributed under the License is distributed on an "AS IS" BASIS,
% WITHOUT WARRANTIES OR CONDITIONS OF ANY KIND, either express or implied.
% See the License for the specific language governing permissions and
% limitations under the License.

\section{\module{e32} ---
  A Symbian OS related services package} 
\label{sec:e32}

\declaremodule{extension}{e32}
\platform{S60}
\modulesynopsis{A Symbian OS related services package.}

The \module{e32} module offers Symbian OS related utilities that are not 
related to the UI and are not provided by the standard Python library 
modules.

\subsection{Module Level Functions}
\label{subsec:e32}

The following free functions - functions that do not belong to any class 
- are defined in the \module{e32} module:

\begin{funcdesc}{ao_yield}{}

Yields to the active scheduler to have ready active objects with priority above 
normal scheduled for running. This has the effect of flushing the eventual 
pending UI events. Note that the UI callback code may be run in the context of 
the thread that performs an \code{ao_yield}. For information on active 
scheduler, see S60 SDK documentation \cite{S60Doc}.

\end{funcdesc}

\begin{funcdesc}{ao_sleep}{interval \optional{, callback}}
Sleeps for the given \var{interval} without blocking the active 
scheduler. When the optional \var{callback} is given, the call 
to \code{ao_sleep} returns immediately and the 
\var{callback} gets called after \var{interval}. See 
also Section \ref{subsec:Aotimer}, Ao_timer Type.
\end{funcdesc}

\begin{funcdesc}{ao_callgate}{wrapped_callable}
Wraps \var{wrapped_callable} into returned callable object 
\var{callgate} that can be called in any thread. As a result of 
a call to \var{callgate}, \var{wrapped_callable} 
gets called in the context of the thread that originally created the 
callgate. Arguments can be given to the call. This is actually a simple 
wrapping of the Symbian active object facility.
\end{funcdesc}

\begin{funcdesc}{drive_list}{}
Returns a list of currently visible drives as a list of Unicode strings 
\code{'<driveletter>:'}
\end{funcdesc}

\begin{funcdesc}{file_copy}{target_name, source_name}
Copies the file \var{source_name} to \var{target_name}. The names must be 
complete paths.
\end{funcdesc}

\begin{funcdesc}{in_emulator}{}
Returns \code{True} if running in an emulator, or \code{False} if running on a 
device.
\end{funcdesc}

\begin{funcdesc}{set_home_time}{time}
Set the device's time to \code{time} (see Section \ref{subsec:datetime}).
\end{funcdesc}

\begin{datadesc}{pys60_version}
A string containing the version number of the Python for S60 and some 
additional information.

Example:
\begin{verbatim}
>>> import e32
>>> e32.pys60_version
'1.9.3 svn2793'
>>>
\end{verbatim}
\end{datadesc}

\begin{datadesc}{pys60_version_info}
A tuple containing the five components of the Python for S60 version
number: major, minor, micro, release tag, and serial. All values
except release level are integers; the release tag is a string. A
value other than \code{'final'} for the release tag signifies a
development release. The \code{pys60_version_info} value corresponding
to the Python for S60 version 1.9.3 is \code{(1, 9, 3, 'svn2793', 0)}.

Example:
\begin{verbatim}
>>> import e32
>>> e32.pys60_version_info
(1, 9, 3, 'svn2793', 0)
>>>
\end{verbatim}
\end{datadesc}

\begin{datadesc}{s60_version_info}
Returns the S60 platform version of the device.

Example:
\begin{verbatim}
>>> import e32
>>> e32.s60_version_info
(3, 0)
>>>
\end{verbatim}
\end{datadesc}

\begin{funcdesc}{is_ui_thread}{}
Returns \code{True} if the code that calls this function runs in the 
context of the UI thread; otherwise returns \code{False}.
\end{funcdesc}

\begin{funcdesc}{start_exe}{filename, command \optional{,wait}}
Launches the native Symbian OS executable \var{filename} 
(Unicode) and passes it the \var{command} string. When 
\var{wait} is set, the function synchronously waits for the exit 
of the executable and returns a value that describes the exit type. Possible 
values are \code{0} for normal exit and \code{2} for abnormal exit.
\end{funcdesc}

\begin{funcdesc}{start_server}{filename}
Starts the Python script in file \var{filename} (Unicode) as a 
server in its own process. Note that \module{appuifw} module is not 
available to a server script.
\end{funcdesc}

\begin{funcdesc}{reset_inactivity}{}
Resets the timers since the user was last active. As a consequence, the device 
backlight is normally turned on when this function is invoked.
\end{funcdesc}

\begin{funcdesc}{inactivity}{}
Returns the time in seconds since the user of the device was last active.
\end{funcdesc}

\begin{funcdesc}{get_capabilities}{}
Returns tuple of capabilities granted to the application.
Example:

\begin{verbatim}
>>> import e32
>>> e32.get_capabilities()
('ReadUserData','WriteUserData')
>>>
\end{verbatim}
\end{funcdesc}

\begin{funcdesc}{has_capabilities}{capability_list}
Check if the application has all the capabilities that are passed in a list, 
'capability_list'. Returns True if all the capabilities specified in the list are 
present, False otherwise.

Examples:

\begin{verbatim}
>>> import e32
>>> e32.has_capabilities(['Location','ReadUserData'])
False
>>> e32.has_capabilities(['Location','SwEvent'])
False
>>> e32.has_capabilities(['WriteUserData','ReadUserData'])
True
>>>
\end{verbatim}
\end{funcdesc}

\subsection{Ao\_lock Type}
\label{subsec:Aolock}

\begin{classdesc}{Ao_lock}{}
Creates an \code{Ao_lock} instance. A Symbian active object based 
synchronization service. This can be used in the main thread without 
blocking the handling of UI events. The application should not exit while a 
thread is waiting in \code{Ao_lock}. If \code{Ao_lock.wait} is called 
while another \code{wait} call is already in progress, an \code{AssertionError} is raised.
\end{classdesc}

Instances of \code{Ao_lock} type have the following methods:

\begin{methoddesc}[Ao_lock]{wait}{}
If the lock has already been signaled, returns immediately. Otherwise blocks 
in wait for the lock to be signaled. Only one waiter is allowed, so you 
should avoid recursive calls to this service. \code{wait} can only be 
called in the thread that created the lock object. During the wait, other 
Symbian-active objects are being served, so the UI will not freeze. This may 
result in the UI callback code being run in the context of the thread that 
is waiting in \code{Ao_lock}. This must be considered when designing 
the application logic.
\end{methoddesc}

\begin{methoddesc}[Ao_lock]{signal}{}
Signals the lock. The waiter is released.
\end{methoddesc}

\subsection{Ao\_timer Type}
\label{subsec:Aotimer}

The rationale for the \code{Ao_timer} type is that you cannot cancel a 
pending \code{e32.ao_sleep}. This is problematic if e.g. the user exits 
an application which is sleeping. In this case a panic would occur since the 
sleep is not cancelled - this is the reason you should avoid using 
\code{e32.ao_sleep} and instead use the \code{Ao_timer} with 
appropriate \code{cancel} calls if there is for example a possibility for 
the user to exit the application during a sleep.

\begin{classdesc}{Ao_timer}{}
Creates an \code{Ao_timer} instance. A Symbian active object based 
sleeping service. This can be used in the main thread without blocking the 
handling of UI events. The application should not exit while a thread has a 
pending \code{after} call in \code{Ao_timer}. Only one \code{after} 
invocation can be pending at time for each instance of this type.
\end{classdesc}

Instances of \code{Ao_timer} type have the following methods:

\begin{methoddesc}[Ao_timer]{after}{interval \optional{,callback}}
Sleeps for the given interval without blocking the active scheduler. When 
the optional callback is given, the call to \code{after} returns 
immediately and the callback gets called after interval.
\end{methoddesc}

\begin{methoddesc}[Ao_timer]{cancel}{}
Cancels a pending \code{after} call.
\end{methoddesc}
