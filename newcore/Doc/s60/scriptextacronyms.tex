% Copyright (c) 2009 Nokia Corporation
%
% Licensed under the Apache License, Version 2.0 (the "License");
% you may not use this file except in compliance with the License.
% You may obtain a copy of the License at
%
%     http://www.apache.org/licenses/LICENSE-2.0
%
% Unless required by applicable law or agreed to in writing, software
% distributed under the License is distributed on an "AS IS" BASIS,
% WITHOUT WARRANTIES OR CONDITIONS OF ANY KIND, either express or implied.
% See the License for the specific language governing permissions and
% limitations under the License.

\section{Acronyms and Abbreviations}
\label{sec:scriptextacronyms}

\begin{table}[htbp]
\begin{center}
\begin{tabular}{l|l}
\hline
{\bf Acronym and meaning} & {\bf Description}  \\
\hline
MMS- Multimedia Messaging Service & Multimedia Messaging Service (MMS) is a new standard in mobile messaging. The difference is that MMS can include not just text, but also sound, images and video.  \\
\hline
Runtimes & Execution environments for applications.  \\
\hline
S60 & A software platform for mobile phones using Symbian Operating System.  \\
\hline
SA- System Attribute & System Attribute.  \\
\hline
SAPI- Service API & A set of language-independent APIs integrated into S60 runtimes (WRT and Flash). These APIs are used to obtain specific information related to platform applications and device data.  \\
\hline
SDK- Software Development Kit & It is a set of programs used by a computer programmer to write application programs.  \\
\hline
SMS- Short Messaging Service & SMS is a service for sending short messages of up to 160 characters (224 characters if using a 5-bit mode) to mobile devices, including cellular phones, smartphones and PDAs.  \\
\hline
URI- Uniform Resource Indicator & Uniform Resource Indicator is a short string of characters that represent the address or location of resources, typically on the internet, and how that resource should be accessed.  \\
\hline
URL- Universal Resource Locator & A specially formatted sequence of characters representing a location on the internet.  \\
\hline
WGS-84- World Geodetic System & The World Geodetic System defines a fixed global reference frame for the Earth, for use in geodesy and navigation. The latest revision is WGS 84 dating from 1984, which will be valid up to about 2010.  \\
\end{tabular}
\caption{Acronyms and Abbreviations}
\end{center}
\end{table}














































