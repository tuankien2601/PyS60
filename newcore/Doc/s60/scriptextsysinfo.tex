% Copyright (c) 2009 Nokia Corporation
%
% Licensed under the Apache License, Version 2.0 (the "License");
% you may not use this file except in compliance with the License.
% You may obtain a copy of the License at
%
%     http://www.apache.org/licenses/LICENSE-2.0
%
% Unless required by applicable law or agreed to in writing, software
% distributed under the License is distributed on an "AS IS" BASIS,
% WITHOUT WARRANTIES OR CONDITIONS OF ANY KIND, either express or implied.
% See the License for the specific language governing permissions and
% limitations under the License.

\section{Sys Info}
\label{sec:scriptextsysinfo}

The SysInfo service provides Read or Write access to system information of a terminal. SysInfo service allows registering to system events identified by System Attributes (SAs). Some of the SAs are modifiable and supports notifications. \break

An Object with an entity and a key represents a System Attribute. An entity broadly represents a component in the device. A key is an attribute of an entity. For example, battery is an entity where, \code{ChargingStatus}, \code{BatteryStrength}, and \code{BatteryLevel} are the Keys of the entity. \break

The following sample code is used to load the provider:

\begin{verbatim}
import scriptext
msg_handle = scriptext.load('Service.SysInfo', 'ISysInfo') 
\end{verbatim}

The following table summarizes the SysInfo Interface:
\begin{table}[htbp]
\begin{center}
\begin{tabular}{l|l}
\hline
{\bf Service provider} & \code{Service.SysInfo}  \\
\hline
{\bf Supported interfaces} & \code{ISysInfo}  \\
\end{tabular}
\end{center}
\end{table}

The following table lists the services available in SysInfo:
\begin{table}[htbp]
\begin{center}
\begin{tabular}{l|l}
\hline
{\bf Services} & {\bf Description} \\
\hline
\code{GetInfo} \ref{subsec:sysinfogetinfo} & Reads system attributes value.  \\
\hline
\code{SetInfo} \ref{subsec:sysinfosetinfo} & Modifies system attributes value.  \\
\hline
\code{GetNotification} \ref{subsec:sysinfogetnotify} & Register for notifications. 
\end{tabular}
\end{center}
\end{table}

\subsection{GetInfo}
\label{subsec:sysinfogetinfo}

\code{GetInfo} retrieves the value of a system attribute. It can be used in both synchronous and asynchronous mode.

The following are the examples for using \code{GetInfo}:

{\bf Synchronous} \break

\begin{verbatim}
self.sysinfo_handle.call("GetInfo", {"Entity": u"General", 
                                     "Key": u"VibraActive", 
                                     "SystemData": {"Status": 1}})
\end{verbatim}

{\bf Asynchronous} \break

\begin{verbatim}
event_id = sysinfo_handle.call("GetInfo", {"Entity": u"Network", 
                                           "Key": u"LocationArea"}, 
                               callback=print_location_area)
\end{verbatim}

where, \code{print_location_area} is user defined function.
\break

The following table summarizes the specification of \code{GetInfo}:
\begin{table}[htbp]
\begin{center}
\begin{tabular}{l|l}
\hline
{\bf Interface} & \code{ISysinfo}  \\
\hline
{\bf Description} & Retrieves the value of a system attribute.  \\
\hline
{\bf Response Model} & Synchronous and asynchronous  \\
\hline
{\bf Pre-condition} &   ISysInfo Interface is loaded.  \\
\hline
{\bf Post-condition} & Returns an object on success.  \\
\end{tabular}
\end{center}
\end{table} 


{\bf Input Parameters} \break

Input parameter specifies the Entity of system attribute information returned.
\begin{table}[htbp]
\begin{center}
\begin{tabular}{l|l|p{4cm}|p{6cm}}
\hline
{\bf Name} & {\bf Type} & {\bf Range} & {\bf Description} \\
\hline
Entity & unicode string & For complete list of supported Entities, refer Key Values \ref{subsec:sysinfokeyval} section. & Entity of system attribute. For example, \code{Battery} \break
\code{Network} and so on.  \\
\hline
Key & unicode string & For complete list of supported Keys, refer Key Values \ref{subsec:sysinfokeyval} section. & Key of system attribute. For example, \code{BatteryStrenth} \break
\code{HomeNetwork} and so on.  \\
\hline
[SystemData] & map & \code{DriveInfo} \break
Drive: unicode string & This is an optional parameter from API definition point of view. For some system attributes, you need to specify input. \break

This map must contain one of the input data specifiers defined in System Data. \break

For more information on input specifier refer the section Key Values \ref{subsec:sysinfokeyval}.  \\
\end{tabular}
\caption{Input parameters for GetInfo}
\end{center}
\end{table}


\newpage
{\bf Output Parameters} \break

Output parameter returns an object that contains the requested information. It also contains \code{ErrorCode} and an \code{ErrorMessage}, if the operation fails.
\begin{table}[htbp]
\begin{center}
\begin{tabular}{l|p{2cm}|p{3cm}|p{3cm}}
\hline
{\bf Name} & {\bf Type} & {\bf Range} & {\bf Description}  \\
\hline
\code{ErrorCode} & int & NA & Contains the SAPI specific error code when the operation fails.  \\
\hline
\code{ErrorMessage} & string & NA & Error Description in Engineering English.  \\
\hline
\code{ReturnValue} & map(System Data) \break
Entity: string \break
Key: string & For complete range of keys for the particular map, refer to {\bf System Data} in {\bf Key Value} section. & Output map always contains Entity and Key. Rest of the elements in the map depends on requested system attribute (Entity-Key). It will be one of the data specifiers defined in System Data. \break

On requesting drive information using system attribute (for example: \code{Memory}, \code{DriveInfo}), \code{ReturnValue} map will contain Keys defined in \code{DriveInfo} Map.   \\
\end{tabular}
\caption{Output parameters for GetInfo}
\end{center}
\end{table}

{\bf Errors} \break

The following table lists the error codes and their values:
\begin{table}[htbp]
\begin{center}
\begin{tabular}{l|l}
\hline
{\bf Error code value} & {\bf Description}  \\
\hline
\code{-304} & General Error  \\
\hline
\code{1002} & Bad argument type  \\
\hline
\code{1003} & Missing argument  \\
\hline
\code{1004} & Service not supported  \\
\hline
\code{1010} & Entry exists  \\
\hline
\code{1012} & Item not found  \\
\end{tabular}
\caption{Error codes}
\end{center}
\end{table}

\newpage
{\bf Error Messages} \break

The following table lists the error messages and their description:
\begin{table}[htbp]
\begin{center}
\begin{tabular}{p{4cm}|p{10cm}}
\hline
{\bf Error messages} & {\bf Description}  \\
\hline
\code{SysInfo:GetInfo: Insufficient Arguments to process} & At least two input arguments are expected to process \code{GetInfo} service request.  \\
\hline
\code{SysInfo:GetInfo: Entity:Input Parameter Missing} & Indicates mandatory parameter Entity is missing in the service request.  \\
\hline
\code{SysInfo:GetInfo: Key:Input Parameter Missing} & Indicates mandatory parameter Key is missing in the service request.  \\
\hline
\code{SysInfo:GetInfo: Incorrect SystemData Type, SystemData Must be a Map} & Indicates that either the optional parameter \code{SystemData} specified is not a map or content of the map is inappropriate to process request.  \\
\hline
\code{SysInfo:GetInfo: CallBack and CmdOptions not matching} & Indicates that the situation where user specified callback and \code{CmdOptions} is set to Synchronous and vice-versa.  \\
\end{tabular}
\caption{Error messages}
\end{center}
\end{table}

\newpage
{\bf Example} \break

The following sample code illustrates how to retrieve the current location area in synchronous mode:

\begin{verbatim}
import scriptext
import e32

# Using e32.Ao_lock() so that the main function can wait 
# till the callback is hit.
lock = e32.Ao_lock()

# Callback function will be called when the requested service is complete
def print_location_area(trans_id, event_id, input_params):
    if event_id != scriptext.EventCompleted:   
# Check the event status
        print "Error in retrieving required info"
        print "Error code is: " + str(input_params["ReturnValue"]["ErrorCode"])
        if "ErrorMessage" in input_params["ReturnValue"]:
            print "Error message is: " + input_params["ReturnValue"]["ErrorMessage"]
    else:
        print "Current Location Area is: " + input_params["ReturnValue"]["Status"]

    lock.signal()

# Load sysinfo service
sysinfo_handle = scriptext.load("Service.SysInfo", "ISysInfo")

# Make a request to query the required information
event_id = sysinfo_handle.call("GetInfo", {"Entity": u"Network", "Key": u"LocationArea"}, callback=print_location_area)

print "Waiting for the request to be processed!"
lock.wait()
print "Request complete!"
\end{verbatim}

\subsection{SetInfo}
\label{subsec:sysinfosetinfo}

\code{SetInfo} modifies the value of a system attribute. It takes a set of input parameters that define entity and key of \code{SystemAttribute} to modify the value of system attribute. \break

It is available in only synchronous mode.

The following is an example for using \code{GetInfo}:

{\bf Synchronous} \break

\begin{verbatim}
sysinfo_handle.call("SetInfo", {"Entity": u"General",
                                "Key":u"VibraActive", 
                                "SystemData" {"Status": 1}}) 
\end{verbatim}

\newpage
The following table summarizes the specification of \code{GetInfo}:
\begin{table}[htbp]
\begin{center}
\begin{tabular}{l|l}
\hline
{\bf Interface} & \code{ISysinfo}  \\
\hline
{\bf Description} & Modifies the value of a system attribute.  \\
\hline
{\bf Response Model} & Synchronous  \\
\hline
{\bf Pre-condition} & ISysInfo Interface is loaded.  \\
\hline
{\bf Post-condition} & Changes the system attribute on success  \\
\end{tabular}
\end{center}
\end{table}

{\bf Input Parameters} \break

Input parameter specifies an entity and key of system attribute.
\begin{table}[htbp]
\begin{center}
\begin{tabular}{l|l|p{4cm}|p{4cm}}
\hline
{\bf Name} & {\bf Type} & {\bf Range} & {\bf Description} \\
\hline
Entity & unicode string & For complete list of supported Entities, refer Key Values \ref{subsec:sysinfokeyval} section. & Entity of system attribute. For example, \code{Connectivity} \break
\code{Display} and so on.  \\
\hline
Key & unicode string & For complete list of supported Keys, refer Key Values \ref{subsec:sysinfokeyval} section. & Key of system attribute. For example, \code{Bluetooth} \break
\code{Wallpaper} and so on.  \\
\hline
SystemData & map & Status information \break
Status: int \break

Wallpaper path \break
\code{StringData}: unicode string & This map must contain one of the input data specifiers defined in System Data. \break

For more information on input specifier refer the section Key Values \ref{subsec:sysinfokeyval}.  \\
\end{tabular}
\caption{Input parameters for SetInfo}
\end{center}
\end{table}

{\bf Output Parameters} \break

Output parameter contains \code{ErrorCode} and an \code{ErrorMessage}, if the operation fails.
\begin{table}[htbp]
\begin{center}
\begin{tabular}{l|l|l|l}
\hline
{\bf Name} & {\bf Type} & {\bf Range} & {\bf Description}  \\
\hline
\code{ErrorCode} & int & NA & Contains the SAPI specific error code when the operation fails.  \\
\hline
\code{ErrorMessage} & string & NA & Error Description in Engineering English.  \\
\end{tabular}
\caption{Output parameters for GetInfo}
\end{center}
\end{table}

\newpage
{\bf Errors} \break

The following table lists the error codes and their values:
\begin{table}[htbp]
\begin{center}
\begin{tabular}{l|l}
\hline
{\bf Error code value} & {\bf Description}  \\
\hline
\code{1002} & Bad argument type  \\
\hline
\code{1003} & Missing argument  \\
\hline
\code{1004} & Service not supported  \\
\hline
\code{1011} & Access denied  \\
\hline
\code{1012} & Item not found  \\
\hline
\code{1014} & General error  \\
\hline
\code{1017} & Path not found  \\
\end{tabular}
\caption{Error codes}
\end{center}
\end{table}

{\bf Error Messages} \break

The following table lists the error messages and their description:
\begin{table}[htbp]
\begin{center}
\begin{tabular}{p{6cm}|p{8cm}}
\hline
{\bf Error messages} & {\bf Description}  \\
\hline
\code{SysInfo:SetInfo: Insufficient Arguments to process} & At least two input arguments are expected to process \code{SetInfo} service request.  \\
\hline
\code{SysInfo:SetInfo: Entity:Input Parameter Missing} & Indicates mandatory parameter Entity is missing in the service request.  \\
\hline
\code{SysInfo:SetInfo: Key:Input Parameter Missing} & Indicates mandatory parameter Key is missing in the service request.  \\
\hline
\code{SysInfo:SetInfo: Incorrect SystemData Type, SystemData Must be a Map} & Indicates that either the optional parameter \code{SystemData} specified is not a map or content of the map is inappropriate to process request.  \\
\hline
\code{SysInfo:SetInfo: SystemData Argument Missing} & Indicates that mandatory parameter SystemData is not specified in input argument list.  \\
\hline
\code{SysInfo:SetInfo: ASync Version Not Supported} & This message is given when \code{SetInfo} is requested by specifying callback or \code{CmdOptions} set to Asynchronous request type.  \\
\end{tabular}
\caption{Error messages}
\end{center}
\end{table}


\newpage
{\bf Example} \break

The following sample code illustrates how to set Vibra mode:

\begin{verbatim}
# Synchronous example: Setting Vibra mode

import scriptext

# Load sysinfo service
sysinfo_handle = scriptext.load('Service.SysInfo', 'ISysInfo')

# Make a request to set vibra mode
try:
    sysinfo_handle.call("SetInfo", {"Entity": u"General", "Key": u"VibraActive", "SystemData": {"Status": 1}})
    print "Request complete!"
except scriptext.ScriptextError:
    print 'Error in servicing the request'
\end{verbatim}

\subsection{GetNotification}
\label{subsec:sysinfogetnotify}

\code{GetNotification} method registers a callback function to receive notifications of system data. It takes a set of input parameters that specifies entity and key of System Attribute. \break
It is available in only asynchronous mode. 

The following is an example for using \code{GetNotification}:

{\bf Asynchronous} \break

\begin{verbatim}
event_id = sysinfo_handle.call("GetNotification", 
                               {"Entity": u"Battery", "Key":  u"ChargingStatus"}, 
                               callback=sysinfo_callback)
\end{verbatim}

where, \code{sysinfo_callback} is user defined function.

The following table summarizes the specification of \code{GetNotification}:
\begin{table}[htbp]
\begin{center}
\begin{tabular}{l|l}
\hline
{\bf Interface} & \code{ISysinfo}  \\
\hline
{\bf Description} & Registers a callback function for listening to notifications.  \\
\hline
{\bf Response Model} & Asynchronous  \\
\hline
{\bf Pre-condition} &   ISysInfo Interface is loaded.  \\
\hline
{\bf Post-condition} & Returns the generic parameter system data on success.  \\
\end{tabular}
\end{center}
\end{table} 

\newpage
{\bf Input Parameters} \break

Input parameter specifies the Entity and Key of system attribute, and system data.
\begin{table}[htbp]
\begin{center}
\begin{tabular}{l|l|p{2cm}|p{6cm}}
\hline
{\bf Name} & {\bf Type} & {\bf Range} & {\bf Description}  \\
\hline
Entity & unicode string & For complete list of supported Entities, refer Key Values \ref{subsec:sysinfokeyval} section. & Entity of system attribute. For example, \code{Battery} \break
\code{Network} and so on.  \\
\hline
Key & unicode string & For complete list of supported Keys, refer Key Values \ref{subsec:sysinfokeyval} section. & Key of system attribute. For example, \code{BatteryStrenth} \break
\code{CurrentNetwork} and so on.  \\
\hline
[SystemData] & map & Status information \break
Status: int \break

\code{DriveInfo} \break
Drive: unicode string \break
CriticalSpace: int & This is an optional parameter from API definition point of view. For some system attributes, you need to specify input. \break

This map must contain one of the input data specifiers defined in System Data. \break

For more information on input specifier refer the section Key Values \ref{subsec:sysinfokeyval}. \break

Here are some system attributes for which status information is used as input specifier. \break

\code{DriveNumber} and critical space to be specified for drive critical memory notifications. \break

For example,
Battery- \code{BatteryStrength} (Threshold Strength value). \break
Network- \code{Signal} (Threshold Signal value).  \\
\end{tabular}
\caption{Input parameters for GetNotification}
\end{center}
\end{table}

\newpage
{\bf Output Parameters} \break

Output parameter returns an object that contains the requested information. It also contains \code{ErrorCode} and an \code{ErrorMessage}, if the operation fails.
\begin{table}[htbp]
\begin{center}
\footnotesize
\begin{tabular}{l|p{2cm}|p{3cm}|p{7cm}}
\hline
{\bf Name} & {\bf Type} & {\bf Range} & {\bf Description}  \\
\hline
\code{ErrorCode} & int & NA & Contains the SAPI specific error code when the operation fails.  \\
\hline
\code{ErrorMessage} & string & NA & Error Description in Engineering English.  \\
\hline
\code{ReturnValue} & map(System Data) \break
Entity: string \break
Key: string & For complete range of keys for the particular map, refer to {\bf System Data} in {\bf Key Value} section. & Output map always contains Entity and Key. Rest of the elements in the map depends on requested system attribute (Entity-Key). It will be one of the data specifiers defined in System Data. \break

On requesting drive information using system attribute (for example: \code{Memory}, \code{DriveInfo}), \code{ReturnValue} map will contain Keys defined in \code{DriveInfo} Map.   \\
\end{tabular}
\caption{Output parameters for GetNotification}
\end{center}
\end{table}

{\bf Errors} \break

The following table lists the error codes and their values:
\begin{table}[htbp]
\begin{center}
\footnotesize
\begin{tabular}{l|l}
\hline
{\bf Error code value} & {\bf Description}  \\
\hline
\code{1002} & Bad argument type  \\
\hline
\code{1003} & Missing argument  \\
\hline
\code{1006} & Service not ready  \\
\hline
\code{1010} & Entry exists  \\
\hline
\code{1012} & Item not found  \\
\end{tabular}
\caption{Error codes}
\end{center}
\end{table}

{\bf Error Messages} \break

The following table lists the error messages and their description:
\begin{table}[htbp]
\begin{center}
\begin{tabular}{p{6cm}|p{8cm}}
\hline
{\bf Error messages} & {\bf Description}  \\
\hline
\code{SysInfo:GetNotification: Insufficient Arguments to process} & At least two input arguments are expected to process \code{GetNotification} service request.  \\
\hline
\code{SysInfo:GetNotification: Entity:Input Parameter Missing} & Indicates mandatory parameter Entity is missing in the service request.  \\
\hline
\code{SysInfo:GetNotification: Key:Input Parameter Missing} & Indicates mandatory parameter Key is missing in the service request.  \\
\hline
\code{SysInfo:GetNotification: Incorrect SystemData Type, SystemData Must be a Map} & Indicates that either the optional parameter \code{SystemData} specified is not a map or content of the map is inappropriate to process request.  \\
\hline
\code{SysInfo:GetNotification: Sync Version Not Supported} & This message is given when \code{GetNotification} is requested without specifying callback or \code{CmdOptions} set to Synchronous request type.  \\
\end{tabular}
\caption{Error messages}
\end{center}
\end{table}

{\bf Example} \break

\begin{verbatim}
import scriptext
import e32

lock = e32.Ao_lock()
messaging_handle = scriptext.load('Service.SysInfo', 'ISysInfo')

def sysinfo_callback(trans_id, event_id, input_params):
    if event_id != scriptext.EventCompleted:   
# Check the event status
        print "Error in retrieving required info"
        print "Error code is: " + str(input_params["ReturnValue"]["ErrorCode"])
        if "ErrorMessage" in input_params["ReturnValue"]:
            print "Error message is: " + input_params["ReturnValue"]["ErrorMessage"]
    else:
        print "Current Battery charging value: " + str(input_params["ReturnValue"]["Status"])
    lock.signal()

# Make a request to get notification
event_id = sysinfo_handle.call("GetNotification", {"Entity": u"Battery", "Key": u"ChargingStatus"}, callback=sysinfo_callback)
lock.wait()
\end{verbatim}

\newpage
\subsection{Key Values}
\label{subsec:sysinfokeyval}

This section details the key values used in the context of SysInfo Service API invocation.

\subsubsection{System Attributes}

{\bf Entity} \break
\begin{table}[htbp]
\begin{center}
\begin{tabular}{l|l|l}
\hline
{\bf Key} & {\bf DataType} & {\bf Description}  \\
\hline
Entity & string & Part of system attribute represents an Entity.  \\
\end{tabular}
\caption{Entity}
\end{center}
\end{table}

{\bf Key} \break
\begin{table}[htbp]
\begin{center}
\begin{tabular}{l|l|l}
\hline
{\bf Key} & {\bf DataType} & {\bf Description}  \\
\hline
Key & string & Part of system attribute represents a key with in Entity.  \\
\end{tabular}
\caption{Key}
\end{center}
\end{table}

\subsubsection{System Data}

\code{SystemData} is a map whose keys are defined by one of the following data specifiers described in this section. These are added to the \code{ReturnValue} map. Typically, \code{SystemData} is status information that is represented using integer. In some cases, it is a map of network details, or a list having connection details of each connection. This section covers all the possible \code{SystemData} types.

\begin{table}[htbp]
\begin{center}
\begin{tabular}{l|l|p{10cm}}
\hline
{\bf Key} & {\bf DataType} & {\bf Description}  \\
\hline
Status & int & This Key provides status information of system attribute. For example, \code{BatteryLevel} (0-7), \code{NetworkMode}, and \code{BTPower} (0-OFF, 1-ON) and so on.  \\
\end{tabular}
\caption{Status}
\end{center}
\end{table}

\begin{table}[htbp]
\begin{center}
\begin{tabular}{l|l|p{10cm}}
\hline
{\bf Key} & {\bf DataType} & {\bf Description}  \\
\hline
\code{StringData} & string & This Key provides data of type string to specify. For example, \code{WallpaperPath}, \code{IMEI Number}, and \code{PhoneModel} and so on.  \\
\end{tabular}
\caption{StringData}
\end{center}
\end{table}

\begin{table}[htbp]
\begin{center}
\begin{tabular}{l|l|p{8cm}}
\hline
{\bf Key} & {\bf DataType} & {\bf Description}  \\
\hline
\code{NetworkName} & string & This Key provides name of the network.  \\
\hline
\code{NetworkStatus} & int & {\bf Status}: {\bf Description} \break
-1: Unknown \break
0: Available. A network that the ME is allowed to register to. \break
1: Current. This is the currently registered network. \break
2: Forbidden. A network that the ME is not allowed to register to.  \\
\hline
\code{NetworkMode} & int & {\bf Mode}: {\bf Description} \break
-1: Unknown \break
0: Unregistered \break
1: Global System for Mobile communications (GSM) \break
2: Advanced Mobile Phone System (AMPS) \break
3: Code Division Multiple Access (CDMA95) \break
4: Code Division Multiple Access (CDMA2000) \break
5: Wideband Code Division Multiple Access (WCDMA) \break
6: Time Division, Code Division Multiple Access (TDCDMA)  \\
\hline
\code{CountryCode} & string & Mobile Country Code (MCC).  \\
\hline
\code{NetworkCode} & string & Mobile Network Code (MNC).  \\
\hline
\code{LocationStatus} & bool & True: Location Area Code (LAC), CellId are valid. \break
False: Location Area Code (LAC), \code{CellId} are invalid.  \\
\hline
\code{AreaCode} & int & Location Area Code LAC.  \\
\hline
\code{CellId} & int & \code{CellId}.  \\
\end{tabular}
\caption{NetworkInfo}
\end{center}
\end{table}

\begin{table}[htbp]
\begin{center}
\begin{tabular}{l|l|p{8cm}}
\hline
{\bf Key} & {\bf DataType} & {\bf Description}  \\
\hline
\code{ConnectionList} & Iterator & This Iterator points to the list of available active data connections. \code{ConnectionInfo} map represents a data connection.  \\
\end{tabular}
\caption{ConnectionList}
\end{center}
\end{table}

\begin{table}[htbp]
\begin{center}
\begin{tabular}{l|l|p{8cm}}
\hline
{\bf Key} & {\bf DataType} & {\bf Description}  \\
\hline
\code{ConnectionStatus} & int & 0: \code{DisConnected} \break
1: \code{Connected}  \\
\hline
\code{IAPID} & int & Access point ID  \\
\hline
\code{ConnectionType} & int & Coloured connection types are not supported. \break
{\bf Mode}: {\bf Description} \break
-1: Unknown \break
0: Circuit Switch Data (CSD) \break
1: WCDMA \break
2: LAN [Emulator] \break
3: CDMA2000 \break
4: General Packet Radio Service (GPRS) \break
5: High Speed Circuit Switched Data (HSCSD) \break
6: Enhanced Data rates for Global Evolution GPRS(EdgeGPRS) \break
7: Wireless Local Area Network(WLAN) \break
8: Bluetooth \break
9: Virtual VPN  \\
\hline
\code{IAPName} & string & Access Point Name. For example, www.airtelgprs.com.  \\
\hline
\code{NetworkName} & string & Network name applicable for WLAN networks.  \\
\hline
\code{IAPConnectionName} & string & The access point connection name that is, \code{MobileOffice}, \code{MyGprs}.  \\
\end{tabular}
\caption{ConnectionInfo}
\end{center}
\end{table}

\begin{table}[htbp]
\begin{center}
\begin{tabular}{l|l|l}
\hline
{\bf Key} & {\bf DataType} & {\bf Description}  \\
\hline
\code{AccessoryType} & int & -1: Unknown \break
0: HeadSet \break
1: BTHeadSet \break
2: CarKit \break
3: BTCarKit  \\
\hline
\code{AccessoryState} & int & -1: Unknown \break
0: Disconnected \break
1: Connected \break
\end{tabular}
\caption{AccessoryInfo}
\end{center}
\end{table}

\begin{table}[htbp]
\begin{center}
\begin{tabular}{l|l|p{10cm}}
\hline
{\bf Key} & {\bf DataType} & {\bf Description}  \\
\hline
\code{AccessoryList} & Iterator & This Iterator points to the list connected accessories. \code{AccessoryInfo} map represents an accessory.  \\
\end{tabular}
\caption{AccessoryList}
\end{center}
\end{table}

\begin{table}[htbp]
\begin{center}
\begin{tabular}{l|l|l}
\hline
{\bf Key} & {\bf DataType} & {\bf Description}  \\
\hline
\code{LanguageList} & List of int & This List points to the list of supported language enumerations, which are defined in S60.  \\
\end{tabular}
\caption{LanguageList}
\end{center}
\end{table}

\begin{table}[htbp]
\begin{center}
\begin{tabular}{l|l|l}
\hline
{\bf Key} & {\bf DataType} & {\bf Description}  \\
\hline
\code{MajorVersion} & string & This List points to the list of supported language enumerations, which are defined in S60.  \\
\hline
\code{MinorVersion} & string & Minor number of the version. For example, for 3.1, \code{1} will be minor.  \\
\end{tabular}
\caption{Version}
\end{center}
\end{table}

\begin{table}[htbp]
\begin{center}
\begin{tabular}{l|l|p{8cm}}
\hline
{\bf Key} & {\bf DataType} & {\bf Description}  \\
\hline
\code{Drive} & string & Drive is a string. For example, {\bf c:>{\textbackslash}{\textbackslash}}, {\bf d:>{\textbackslash}{\textbackslash}} and so on.  \\
\hline
\code{TotalSpace} & string & Total Space in bytes.  \\
\hline
\code{FreeSpace} & string & Free Space in bytes.  \\
\hline
\code{CriticalSpace} & int & This is the critical free space in bytes.  \\
\hline
\code{MediaType} & int & 0: \code{MediaNotPresent} \break
1: \code{MediaUnknown} \break
2: \code{MediaFloppyDisk} \break
3: \code{MediaHardDisk} \break
4: \code{MediaCdRom} \break
5: \code{MediaRam} \break
6: \code{MediaFlash} \break
7: \code{MediaRom} \break
8: \code{MediaRemote} \break
9: \code{MediaNANDFlash} \break
10: \code{MediaRotatingMedia}  \\
\hline
\code{BatteryState} & int & 0: \code{BatNotSupported} \break
1: \code{BatGood} \break
2: \code{BatLow}  \\
\hline
\code{DriveName} & string & Drive name  \\
\end{tabular}
\caption{DriveInfo}
\end{center}
\end{table}

\begin{table}[htbp]
\begin{center}
\begin{tabular}{l|l|l}
\hline
{\bf Key} & {\bf DataType} & {\bf Description}  \\
\hline
XPixels & int & X-Pixels  \\
\hline
YPixels & int & Y-Pixels  \\
\end{tabular}
\caption{Resolution}
\end{center}
\end{table}

\begin{table}[htbp]
\begin{center}
\begin{tabular}{l|l|p{8cm}}
\hline
{\bf Key} & {\bf DataType} & {\bf Description}  \\
\hline
\code{DriveList} & List of strings & This List points to the list of drives in the terminal. Drives are represented as strings. For example, {\bf c:>{\textbackslash}{\textbackslash}} and so on.  \\
\end{tabular}
\caption{DriveList}
\end{center}
\end{table}

\begin{table}[htbp]
\begin{center}
\begin{tabular}{l|l|l}
\hline
{\bf Key} & {\bf DataType} & {\bf Description}  \\
\hline
\code{StringList} & List of strings & This List points to the list of available USB modes.  \\
\end{tabular}
\caption{StringList}
\end{center}
\end{table}

\pagebreak
