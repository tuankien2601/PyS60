% Copyright (c) 2009 Nokia Corporation
%
% Licensed under the Apache License, Version 2.0 (the "License");
% you may not use this file except in compliance with the License.
% You may obtain a copy of the License at
%
%     http://www.apache.org/licenses/LICENSE-2.0
%
% Unless required by applicable law or agreed to in writing, software
% distributed under the License is distributed on an "AS IS" BASIS,
% WITHOUT WARRANTIES OR CONDITIONS OF ANY KIND, either express or implied.
% See the License for the specific language governing permissions and
% limitations under the License.

\section{Overview of scriptext usage}
\label{sec:scriptextpyruntime}

The following section describes the usage of scriptext module in detail.

\subsection{Module level functions and Data Types}
\label{subsec:pydatatypes}

{\bf Load} \break

This is the only module level function. It can be used to load a particular service provider. It returns a handler object that can be used to invoke further services supported by the provider.

{\bf scriptextHandle} \break

This is the type of the object which is returned by \code{Load()} API. Services can be invoked using the \code{Call()} API of this object.

The provider returns data wrapped in the objects as mentioned in the following:

\begin{notice}[note]
Examples of using these objects are listed in the following sections.
\end{notice}

{\bf scriptextmap} \break

Python wrapper over Platform Service API map object. The following operations are supported over this object:

\begin{itemize}
\item Finding length of the map
\item Getting value with a key
\item Finding if a key exists in the map
\item Iterating over it
\end{itemize}

{\bf scriptextlist} \break

Python wrapper over Platform Service API List object. The following operations are supported over this object:

\begin{itemize}
\item Finding its length
\item Getting value at a given index
\item Iterating over it
\end{itemize}

{\bf scriptextiterable} \break

Python wrapper over Platform Service API Iterable object. The following operation is supported over this object:

\begin{itemize}
\item Iterating over it
\end{itemize}

\subsection{Instantiating a Service Object}
\label{subsec:pyservice}

A Service object can be instantiated using the \code{load} API, with the required service provider details. Import scriptext module before using the method.

{\bf Syntax} \break

\begin{verbatim}
import scriptext
scriptext_handle = scriptext.load(<provider>, <interface>)
\end{verbatim}

{\bf Arguments} \break

The first argument is a string that specifies the name of the service provider. \break

The second argument is a string, that specifies one of the supported interfaces of the <provider>.

{\bf Return Value} \break

The load() method returns a service object for a successful call. This method raises \code{ScriptextError} when the service provider name or the interface name is null.

{\bf Example} \break

The following sample code illustrates how to instantiate messaging service object:

\begin{verbatim}
import scriptext

try:
    messaging_handle = scriptext.load("Service.Messaging", "IMessaging")

except err:
    # Handle error while instantiating the service.
\end{verbatim}

\subsection{Making Synchronous Request}
\label{subsec:pyrequestsyncservice}

This method is used to request a specific synchronous service or operation from a service provider, using \code{call()} API.

{\bf Syntax} \break

\begin{verbatim}
result = service_instance_object.call(operation, parameters)
\end{verbatim}

{\bf Arguments} \break

The operation argument describes the service requested from the service provider. \break

The parameters argument is a dictionary, which specifies input parameters to the specified request. \break

{\bf Return Value} \break

The Return value contains the service output, and its data type depends on the service requested.

{\bf Example} \break

The following sample code illustrates how to retrieve all the Sender \code{IDs} from Inbox using \code{GetList}:

\begin{verbatim}
import scriptext

messaging_handle = scriptext.load('Service.Messaging', 'IMessaging')

# This 'GetList' request returns all the SMS in the inbox as an iterable map

sms_iter = messaging_handle.call('GetList', {'Type': u'Inbox'})

sender_list = [] 

for sms_dict in sms_iter: 
 if sms_dict['MessageType'] == 'SMS': 
  sender_list.append(sms_dict['Sender'])

print "ID list :", sender_list
\end{verbatim}

\subsection{Making Asynchronous Request}
\label{subsec:pyrequestasyncservice}

For making an asynchronous request, a call back function needs to be defined and passed as an additional parameter to the \code{call()} API.

{\bf Syntax} \break

\begin{verbatim}
result = service_instance_object.call(operation, parameters, callback=callback_function)
\end{verbatim}

{\bf Arguments} \break

The operation argument describes the service requested from the service provider. The parameters argument is a dictionary, which specifies input parameters to the specified request. \break

\code{callback_function} is an user defined callback function.

The following sample code illustrates how to define a callback handler function to handle the response from an asynchronous request:

\begin{verbatim}
def callback_function(transactionID, eventID, outParam)
\end{verbatim}

The following table describes the arguments of the call back function:

\break
\begin{table}[htbp]
\begin{center}
\begin{tabular}{ p{3cm}| p{6cm}| p{6cm} |}
\hline
{\bf Argument} & {\bf Description} & {\bf Value}  \\
\hline
\code{transactionID} & This is the unique transaction ID associated with the particular asynchronous request. & It is returned as part of the result of the initial asynchronous call. \\
\hline
\code{eventID} & Specifies the asynchronous operation status. & For a complete list of \code{EventID}, see \code{EventID} \ref{subsec:appenevent} section in the {\bf Appendix}.  \\
\hline
\code{outParam} & This argument is a dictionary that holds the output of an asynchronous call. & Refer to the following table for the dictionary items in \code{outParam}.  \\
\end{tabular}
\end{center}
\end{table}

The \code{outParam} argument of callback method is a map containing the return value, an error code, and an error message.

\begin{table}[htbp]
\begin{center}
\begin{tabular}{ l | p{6cm}| p{6cm} |}
\hline
{\bf Properties} & {\bf Description} & {\bf Values}  \\
\hline
\code{ReturnValue} & This key contains the information requested by the asynchronous call that initiated the callback. \break

This key is present only if the requested service has a value to return. In this case, \code{outParam} contains only \code{ErrorCode} and \code{ErrorMessage}. & Depends on the Platform Service API and the asynchronous method that was called. Not all calls return this property.  \\
\hline
\code{ErrorCode} & Specifies a pre-defined error code & For detail information about Platform Service API error codes and their descriptions, see Service API Error Codes and Description \ref{subsec:appenerrors} section in the {\bf Appendix}.  \\
\hline
\code{ErrorMessage} & Describes the error & Depends on the Platform Service API and the asynchronous method that is called.  \\
\end{tabular}
\end{center}
\end{table}

\break
{\bf Example} \break

The following sample code illustrates how to retrieve media files from a database, using the operation:

\begin{verbatim}
import scriptext
import e32

def media_callback(trans_id, event_id, output_params):
    # Check if we are interested in this transaction
    if trans_id == media_trans_id:
        print "Not the transaction in which we are interested!"
        return

    # Check if the transaction is complete
    if event_id != scriptext.EventCompleted:
        print "Transaction not complete!"
        return

    # Check if the transaction has resulted in any error
    if output_params['ReturnCode'] != 0:
        print output_params['ReturnMessage']
    else:
        song_list = []
        for item in output_params['ReturnValue']: 
            song_list.append(item['FileName'])
        print "List of files retrieved:", song_list

    lock.signal()


lock = e32.Ao_lock()
media_handle = scriptext.load('Service.MediaManagement', 'IDataSource')

# Request for the list of mp3s in ascending order

media_trans_id = media_handle.call('GetList', 
                                   {'Type': u'FileInfo': u'FileExtension', 
                                               'StartRange': u'.mp3'},
                                    'Sort': {'Key': u'FileName', 'Order': u'Ascending'}}, 
                                   callback=media_callback)

lock.wait()
\end{verbatim}

\subsection{Cancelling of Asynchronous Service Request}
\label{subsec:pycancelasyncservice}

To cancel an asynchronous request, \code{Cancel} is passed as the operation argument in the \code{call()} API. The \code{transactionID} associated with the asynchronous operation also needs to be passed. After completing the cancel operation, the callback function is called with \code{Event_id} as \code{scriptext.EvenCanceled}.

{\bf Syntax} \break

\begin{verbatim}
serviceInstance.call('Cancel', {'TransactionID': serviceTransactionID})
\end{verbatim}

where, \code{transactionID} is associated with the asynchronous operation, which needs to be cancelled.

{\bf Example} \break

The following sample code illustrates how to send and cancel an SMS in asynchronous mode:

\begin{verbatim}
import scriptext
messaging_handle = scriptext.load('Service.Messaging', 'IMessaging')

def sms_send_callback(trans_id, event_id, output_params):
    if sms_trans_id == trans_id:
        if event_id == scriptext.EventCanceled:
            print "SMS Send Canceled"
        else:
            print "Event_id was not scriptext.EventCanceled"
    else:
        print "Invalid transaction ID received"

sms_trans_id = messaging_handle.call('Send', {'MessageType': u'SMS', 
                                              'To': u'12345678', 'BodyText': u'Hi'}, 
                                     callback=sms_send_callback)
try:
    messaging_handle.call('Cancel', {'TransactionID': sms_trans_id})
except scriptext.ScriptextError, err:
    print "Error cancelling request ", err
\end{verbatim}

\pagebreak


